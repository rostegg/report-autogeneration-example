\documentclass[a4paper]{article}
\usepackage[14pt]{extsizes}
\usepackage[utf8]{inputenc}
\usepackage[english,american,ukrainian]{babel}
\usepackage{setspace,amsmath}
\usepackage{kantlipsum}
\usepackage[left=20mm, top=15mm, right=15mm, bottom=15mm, nohead, footskip=10mm]{geometry}
\usepackage{enumitem}
\setlist{leftmargin=15.5mm}
\usepackage{graphicx}
\pagestyle{empty}
\graphicspath{ {./images/} }
\usepackage{amsmath}
\usepackage{makecell}
\usepackage{times} 
\renewcommand\theadalign{bc}
\renewcommand\theadfont{\bfseries}
\renewcommand\theadgape{\Gape[4pt]}
\renewcommand\cellgape{\Gape[4pt]}
 
\begin{document}
 
\begin{center}
\hfill \break
\large{МІНІСТЕРСТВО ОСВІТИ ТА НАУКИ УКРАЇНИ}\\
\large{НАЦІОНАЛЬНИЙ ТЕХНІЧНИЙ УНІВЕРСИТЕТ УКРАЇНИ}\\
\large{''КИЇВСЬКИЙ ПОЛІТЕХНІЧНИЙ ІНСТИТУТ ІМЕНІ ІГОРЯ СІКОРСЬКОГО''}\\

\hfill \break
\normalsize{Кафедра автоматики та управління в технічних системах}\\
\hfill\break
\hfill \break
\hfill \break
\hfill \break
\large{''Методи оптимізації в керуванні та управлінні–2.}\\
\large{Прикладні задачі оптимального керування''}\\
\hfill \break
\hfill \break
\hfill \break
\normalsize{АКОР у задачі регулятора стану}\\
\hfill \break
\hfill \break
\hfill \break
\hfill \break
\end{center}
\hfill \break
\hfill \break
\begin{flushright} 
Виконав ст. гр. IA-82мп\\
{{studentName}}\\
\hfill \break
\hfill \break
Перевірив доц. каф. АУТС\\
Писаренко А.В.\\
\end{flushright}
\hfill \break
\hfill \break
\hfill \break
\hfill \break
\hfill \break
\hfill \break
\begin{center} Київ 2019 \end{center}
 
% title end
 
\newpage
Тема: АКОР у задачі регулятора стану\\
\hfill \break
Порядок виконання\\
\hfill \break
\indent 1) Від заданої передавальної функції об’єкту керування перейти до
векторно-матричної моделі (будь-яким методом).\\

\hfill \break
\indent 2) Дослідити керованість об’єкту.\\
\hfill \break
\indent 3) Розв’язати аналітично рівняння Ріккаті \\
\hfill \break
\indent 4) Виконати перевірку отриманих аналітично результатів за допомогою
вбудованих можливостей MATLAB. Визначити вектор зворотних зв’язків за
станами за допомогою команди \\
\hfill \break
\indent 5) Побудувати модель оптимальної системи у MATLAB/Simulink\\
\hfill \break
\indent 6) Дослідити, чи досягає критерій якості мінімального значення,
заповнивши таблицю\\
\hfill \break
\indent 7) Зробити висновки.\\

Хід роботи\\

\hfill \break
\indent 1. Перейдемо до векторно-матричної моделі за допомогою пакету
MATLAB\\

\begin{center}
\includegraphics[width=0.5\textwidth]{functionInit}
\end{center}

\[
A =
\begin{bmatrix}
    {{a11}} & {{a12}} \\
    {{a21}} & {{a22}} \\
\end{bmatrix},
B =
\begin{bmatrix}
    {{b11}} \\ {{b21}}
\end{bmatrix}
\]

\[
C =
\begin{bmatrix}
    {{c11}} & {{c12}} \\
\end{bmatrix},
D =
\begin{bmatrix}
    0
\end{bmatrix}
\]

\hfill \break
\indent 2. Перевіримо керованість об’єкту\\

\[
P = \big[ B \; AB \big] = 
\left[ 
  \begin{array}{cc}

\begin{matrix}
{{b11}} \\
{{b21}} \\
\end{matrix}
 & \begin{bmatrix}
{{a11}} & {{a12}} \\
{{a21}} & {{a22}}
\end{bmatrix} * 
\begin{bmatrix}
{{b11}} \\ {{b21}}
\end{bmatrix}
\end{array}\right]
=
\begin{bmatrix}
{{p11}} & {{p12}} \\
{{p21}} & {{p22}}
\end{bmatrix}
\]

\hfill \break
\indent \indent n=rank(P)\\
\hfill \break
\indent \indent n=2 - система керована\\

\hfill \break
\indent 3. Аналітичне визначення рівняння Ріккаті\\

\[
Q =
\begin{bmatrix}
    {{originalQ11}} & 0 \\
    0 & {{originalQ22}} \\
\end{bmatrix}\\
\]
\[
R = {{originalR}}
\]

\[
-\hat{\boldsymbol{K}}\boldsymbol{A} - \boldsymbol{A}^T\hat{\boldsymbol{K}} + \hat{\boldsymbol{K}}\boldsymbol{B}\boldsymbol{r}^{-1}\boldsymbol{B}^T\hat{\boldsymbol{K}}-\boldsymbol{Q} = 0
\]

\[
-\hat{\boldsymbol{K}}\boldsymbol{A} =
-\begin{bmatrix}
    k11 & k12 \\
    k21 & k22 \\
\end{bmatrix}\\ *
\begin{bmatrix}
    {{a11}} & {{a12}} \\
    {{a21}} & {{a22}} \\
\end{bmatrix}\\ = 
\begin{bmatrix}
    {{temporaryFirst11}} & {{temporaryFirst12}} \\
    {{temporaryFirst21}} & {{temporaryFirst22}} \\
\end{bmatrix}\\
\]

\[
-\boldsymbol{A}^T\hat{\boldsymbol{K}} =
-\begin{bmatrix}
    {{a11}} & {{a21}} \\
    {{a12}} & {{a22}} \\
\end{bmatrix}\\ *
\begin{bmatrix}
    k11 & k12 \\
    k21 & k22 \\
\end{bmatrix}\\ = 
\begin{bmatrix}
    {{temporarySecond11}} & {{temporarySecond12}} \\
    {{temporarySecond21}} & {{temporarySecond22}} \\
\end{bmatrix}\\
\]

\[
\hat{\boldsymbol{K}}\boldsymbol{B}\boldsymbol{r}^{-1}\boldsymbol{B}^T\hat{\boldsymbol{K}} =
\begin{bmatrix}
    k11 & k12 \\
    k21 & k22 \\
\end{bmatrix}\\ *
\begin{bmatrix}
    {{b11}} \\
    {{b21}} \\
\end{bmatrix}\\ *
{{originalR}}^{-1} *
\begin{bmatrix}
    {{b11}} &&
    {{b21}} 
\end{bmatrix}\\
\begin{bmatrix}
    k11 & k12 \\
    k21 & k22 \\
\end{bmatrix}\\ = 
\]

\[
\begin{bmatrix}
    {{temporaryThird11}} & {{temporaryThird12}} \\
    {{temporaryThird21}} & {{temporaryThird22}} \\
\end{bmatrix}\\
\]


% here maybe problems with align
\[
\begin{bmatrix}
    {{temporaryFirst11}} & {{temporaryFirst12}} \\
    {{temporaryFirst21}} & {{temporaryFirst22}} \\
\end{bmatrix}\\+
\begin{bmatrix}
    {{temporarySecond11}} & {{temporarySecond12}} \\
    {{temporarySecond21}} & {{temporarySecond22}} \\
\end{bmatrix}\\+
\]
\[
\begin{bmatrix}
    {{temporaryThird11}} & {{temporaryThird12}} \\
    {{temporaryThird21}} & {{temporaryThird22}} \\
\end{bmatrix}\\-
\begin{bmatrix}
    {{originalQ11}} & 0 \\
    0 & {{originalQ22}} \\
\end{bmatrix}\\ = 0
\]
\[
\begin{bmatrix}
    {{riccatiMatrix11}} & {{riccatiMatrix12}}\\
    {{riccatiMatrix21}} & {{riccatiMatrix22}} \\
\end{bmatrix}\\ = 0
\]

\begin{center}
\hfill \break
\big[ k11, k12, k21, k22 \big] = solve('${{riccatiMatrix11}}$', \\
'${{riccatiMatrix12}}$',\\
'${{riccatiMatrix21}}$',\\
'${{riccatiMatrix22}}$')\\
\end{center}
\hfill \break
\indent З декількох отриманих розв’язків для кожного коефіцієнту вибрати
один, що відповідає властивості матриці , застосовуючи команди:\\
\hfill \break
\indent >> eig ([k11(1) k12(1); k21(1) k22(1)]) \\
\indent ans = \\
\indent {{eigResult11}} \\
\indent {{eigResult12}} \\
\indent >> eig ([k11({{eigResultTrueIndex}}) k12({{eigResultTrueIndex}}); k21({{eigResultTrueIndex}}) k22({{eigResultTrueIndex}})]) \\
\indent ans = \\
\indent {{eigResultTrue1}} \\
\indent {{eigResultTrue2}} \\
\hfill \break
\indent Визначили що {{eigResultTrueIndex}}-й набір кращий
\[
\hat{\boldsymbol{K}} =
\begin{bmatrix}
    {{riccatiResult11}} & {{riccatiResult12}}\\
    {{riccatiResult21}} & {{riccatiResult22}} \\
\end{bmatrix}\\
\]


\hfill \break
\indent Розрахувати вектор зворотних зв’язків за станами\\

\[
G = r^{-1}B^{T}\hat{\boldsymbol{K}} =
{{originalR}}^{-1} *
\begin{bmatrix}
    {{b11}} & {{b21}}
\end{bmatrix}\\ *
\begin{bmatrix}
    {{riccatiResult11}} & {{riccatiResult12}}\\
    {{riccatiResult21}} & {{riccatiResult22}} \\
\end{bmatrix}\\ =
\begin{bmatrix}
    {{backPropG11}} & {{backPropG12}}\\
\end{bmatrix}\\
\]


\hfill \break
\indent 4. Виконаємо перевірку командою\\
\begin{center}
g = lqr(A, B, Q , R)\\
\hfill \break
>> g = lqr(A, B, Q , r)\\
\hfill \break
g = \big[ {{backPropG11}} {{backPropG12}} \big] \\
\hfill \break
\end{center}

\hfill \break
\indent 5. Побудуємо модель оптимальної системи у MATLAB/Simulink\\
\begin{center}
\includegraphics[width=0.5\textwidth]{simulinkModel}
\end{center}
\begin{center}
Рисунок 1 – Схема визначення екстремуму та тривалості перехідного процесу\\
\end{center}
\begin{center}
\includegraphics[width=0.5\textwidth]{simulinkPlot}
\end{center}
\begin{center}
Рисунок 2 – Графік змінних стану\\
\end{center}

\hfill \break
\indent 6. Дослідити, чи досягає критерій якості мінімального значення

\begin{center}
Таблиця 1 - Критерій якості мінімального значення\\
\end{center}

\begin{center}
\begin{tabular}{ |p{3cm}||p{3cm}|p{3cm}| }
 \hline
 \multicolumn{2}{|c|}{Змінюємо} & Отримуємо\\
 \hline
 $|g_{1}|$ & $|g_{2}|$ & J \\
 \hline
 {{table1Row1g1}} & {{table1Row1g2}} & {{table1Row1J}} \\
 {{table1Row2g1}} & {{table1Row2g2}} & {{table1Row2J}} \\
 {{table1Row3g1}} & {{table1Row3g2}} & {{table1Row3J}} \\
 {{table1Row4g1}} & {{table1Row4g2}} & {{table1Row4J}} \\
 {{table1Row5g1}} & {{table1Row5g2}} & {{table1Row5J}} \\
 {{table1Row6g1}} & {{table1Row6g2}} & {{table1Row6J}} \\
 {{table1Row7g1}} & {{table1Row7g2}} & {{table1Row7J}} \\
 {{table1Row8g1}} & {{table1Row8g2}} & {{table1Row8J}} \\
 {{table1Row9g1}} & {{table1Row9g2}} & {{table1Row9J}} \\
 \hline
\end{tabular}
\end{center}

\begin{center}
Таблиця 2 - Критерій якості мінімального значення\\
\end{center}

\begin{center}
\begin{tabular}{ |p{3cm}||p{3cm}||p{3cm}||p{3cm}||p{3cm}| }
 \hline
 \multicolumn{2}{|c|}{Змінюємо} & Розрахувати & \multicolumn{2}{|c|}{Отримати з моделі}\\
 \hline
 q11,q22 & r & $[g1,g2]$ & J & T\\
 \hline
 Задане & Задане & \makecell{ {{table2Row1g1}} \\ {{table2Row1g2}} } & {{table2Row1J}} & {{table2Row1T}} \\
 > & Задане & \makecell{ {{table2Row2g1}} \\ {{table2Row2g2}} } & {{table2Row2J}} & {{table2Row2T}} \\
 < & Задане & \makecell{ {{table2Row3g1}} \\ {{table2Row3g2}} } & {{table2Row3J}} & {{table2Row3T}} \\
 Задане & > & \makecell{ {{table2Row4g1}} \\ {{table2Row4g2}} } & {{table2Row4J}} & {{table2Row4T}} \\
 Задане & < & \makecell{ {{table2Row5g1}} \\ {{table2Row5g2}} } & {{table2Row5J}} & {{table2Row5T}} \\
 > & > & \makecell{ {{table2Row6g1}} \\ {{table2Row6g2}} } & {{table2Row6J}} & {{table2Row6T}} \\
 < & < & \makecell{ {{table2Row7g1}} \\ {{table2Row7g2}} } & {{table2Row7J}} & {{table2Row7T}} \\
 > & < & \makecell{ {{table2Row8g1}} \\ {{table2Row8g2}} } & {{table2Row8J}} & {{table2Row8T}} \\
 < & > & \makecell{ {{table2Row9g1}} \\ {{table2Row9g2}} } & {{table2Row9J}} & {{table2Row9T}} \\
 \hline
\end{tabular}
\end{center}

\begin{center}
Висновки \\
\end{center}

\indent На даній розрахунковій роботі було досліджено АКОР у задачі
регулятора стану. Результати отримані аналітично та за допомогою
матлаб співпадають. Дослідження зміни критерію за рахунок зміни
коефіцієнтів вектору зворотних зв’язків показали, що результат
отриманий під час розрахунків дає найкращий критерій. Було
проведено дослідження впливу коефіцієнтів матриць штрафів на
критерій та на час перехідного процесу. \\
\indent Експериментуючи зі зміною матриць штрафів Q та r показали:

\begin{itemize}
  \item Збільшення або зменшення матриць штрафів пропорційно впливає на
зміну критерію та не впливає на час п.п.
  \item Збільшення Q та зменшення r призведе до погіршення критерію, проте
зменшить час пп.
  \item При збільшенні лише матриці Q збільшиться критерій та зміниться час
пп та навпаки при зменшенні.
  \item Зміна матриці r призводить до аналогічних змін критерію та часу
перехідного процесу.
\end{itemize}

\newpage
 
\end{document}